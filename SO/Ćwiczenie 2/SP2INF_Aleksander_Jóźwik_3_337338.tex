\documentclass[]{article}

% =============================================================================
% DOCUMENT SETUP & LANGUAGE
% =============================================================================

\usepackage{polyglossia}
\setmainlanguage{polish}

% =============================================================================
% FONT
% =============================================================================
\usepackage{fontspec}
\newfontfamily{\nerdfont}{Arimo Nerd Font}
\setsansfont{Arimo Nerd Font}
\setmonofont{Iosevka Nerd Font}
\renewcommand{\familydefault}{\sfdefault}

% =============================================================================
% COLORS
% =============================================================================
\usepackage{xcolor} % Driver for color support
\definecolor{accent}{HTML}{8839ef} % Main accent color
\definecolor{codegray}{rgb}{0.95,0.95,0.95} % Background for code blocks
\definecolor{commentgreen}{rgb}{0,0.4,0}    % Color for comments in code
\definecolor{keywordblue}{rgb}{0,0,0.6}     % Color for keywords in code
\definecolor{sysop}{HTML}{ed7c50}
\definecolor{fizyka}{HTML}{8839ef}
\definecolor{jimp}{HTML}{dd7878}
\definecolor{podstawyinformatyki}{HTML}{1e66f5}

% =============================================================================
% PAGE LAYOUT
% =============================================================================
\usepackage[a4paper, margin=2.5cm]{geometry} % Page dimensions
\usepackage{titlesec} % For customizing section titles

% Adjust spacing for section titles
\titlespacing*{\section}{0pt}{1ex}{1ex}
\titlespacing*{\subsection}{0pt}{0.5ex}{0.5ex}

% =============================================================================
% GRAPHICS, FIGURES & BOXES
% =============================================================================
\usepackage{graphicx}   % For including images
\usepackage{float}      % Improved float control (e.g., [H] placement)
\usepackage[most]{tcolorbox} % For creating colored boxes
\usepackage{subcaption} % For subfigures

% =============================================================================
% TABLES
% =============================================================================
\usepackage{tabularx}   % For tables with adjustable-width columns
\usepackage{colortbl}   % For colored table cells
\usepackage{booktabs}   % For high-quality table rules
\usepackage{array}      % Enhanced array and tabular environments

% Adjust table row height and column separation
\renewcommand{\arraystretch}{1.25} 
\setlength{\tabcolsep}{10pt}

% =============================================================================
% CODE LISTINGS
% =============================================================================
\usepackage{listings} % For typesetting code

\lstdefinestyle{mystyle}{
	language=c,
	commentstyle=\color{commentgreen},
	keywordstyle=\color{accent},
	numberstyle=\tiny\color{gray},
	stringstyle=\color{purple},
	basicstyle=\ttfamily\footnotesize,
	breakatwhitespace=false,
	breaklines=true,
	captionpos=b,
	keepspaces=true,
	numbers=left,
	numbersep=5pt,
	showspaces=false,
	showstringspaces=false,
	showtabs=false,
	tabsize=4
}

% 2. Zdefiniuj nowy styl pudełka dla kodu
\newtcblisting{mylisting}{
	arc=2mm,
	top=0mm,
	bottom=0mm,
	left=5mm,
	right=0mm,
	boxrule=0.5pt,
	colback=codegray,
	colframe=accent,
	listing only,
	listing options={style=mystyle},
}
% =============================================================================
% BIBLIOGRAPHY & HYPERLINKS
% =============================================================================
\usepackage[backend=biber]{biblatex}
\addbibresource{bibliography.bib} % Bibliography database file

\usepackage{hyperref} % For hyperlinks and PDF metadata
\usepackage{xurl}     % For handling long URLs

\hypersetup{
	colorlinks=true,
	allcolors=black % Set all link colors to black
}

% =============================================================================
% CUSTOM COMMANDS & ENVIRONMENTS
% =============================================================================

% --- Custom Section Boxes ---
% Uses tcolorbox to create styled boxes for section titles.
\newcommand{\mysectionbox}[1]{%
	\begin{tcolorbox}[
		enhanced, colback=accent, colframe=accent, boxrule=1pt, arc=2mm
		]
		\color{white}\thesection\hspace{1em}#1
	\end{tcolorbox}%
}

\newcommand{\mysubsectionbox}[1]{%
	\begin{tcolorbox}[
		enhanced, colback=accent!60, colframe=accent!60, arc=2mm, boxrule=1pt
		]
		\color{black}\thesubsection\hspace{1em}#1
	\end{tcolorbox}%
}

\newcommand{\mysubsubsectionbox}[1]{%
	\begin{tcolorbox}[
		enhanced, colback=accent!20, colframe=accent!20, arc=2mm, boxrule=1pt
		]
		\color{black}\thesubsubsection\hspace{1em}#1
	\end{tcolorbox}%
}

% Box for unnumbered sections (e.g., Bibliography)
\newcommand{\mysectionboxstar}[1]{
	\begin{tcolorbox}[
		enhanced, colback=accent, colframe=accent, boxrule=1pt, arc=2mm
		]
		\color{white}#1
	\end{tcolorbox}%
}

% --- Framed Figures ---
% Creates a figure with a colored frame and shadow.
\newcommand{\framedfigure}[4]{%
	\begin{figure}[H]
		\centering
		\begin{tcolorbox}[
			enhanced, colframe=accent, colback=white, boxrule=1.5pt,
			sharp corners, boxsep=0pt, left=0pt, right=0pt, top=0pt, bottom=0pt,
			hbox, drop shadow
			]
			\includegraphics[width=#2]{#1}
		\end{tcolorbox}
		\caption{#3}
		\label{#4}
	\end{figure}
}

% Creates a subfigure with a colored frame and shadow.
\newcommand{\framedsubfigure}[4]{%
	\begin{subfigure}{#2}
		\centering
		\begin{tcolorbox}[
			enhanced, colframe=accent, colback=white, boxrule=1.5pt,
			sharp corners, boxsep=0pt, left=0pt, right=0pt, top=0pt, bottom=0pt,
			hbox, drop shadow
			]
			\includegraphics[width=\linewidth]{#1}
		\end{tcolorbox}
		\caption{#3}
		\label{#4}
	\end{subfigure}
}

% =============================================================================
% TITLE FORMATTING
% =============================================================================
% Apply the custom boxes to section titles.
\titleformat{\section}{\normalfont}{}{0pt}{\mysectionbox}[]
\titleformat{\subsection}{\normalfont}{}{0pt}{\mysubsectionbox}[]
\titleformat{\subsubsection}{\normalfont}{}{0pt}{\mysubsubsectionbox}[]
\titleformat{name=\section, numberless}{\normalfont}{}{0pt}{\mysectionboxstar}[]

% =============================================================================
% OTHER SETTINGS
% =============================================================================
\let\oldtextbf\textbf
\renewcommand{\textbf}[1]{%
	\textcolor{accent}{\oldtextbf{#1}}%
}

% Author icon
\newcommand{\authoricon}{{\nerdfont\textcolor{accent}{ \space}}}

% Date icon
\newcommand{\dateicon}{{\nerdfont\textcolor{accent}{ \space}}}

\definecolor{accent}{HTML}{ed7c50}
\title{Ćwiczenie 2. Badanie usług katalogowych eDirectory w oparciu o oprogramowanie klienta sieci OES zainstalowanego w systemie Microsoft Windows}
\author{\authoricon Aleksander Jóźwik}
\date{\dateicon \today}

\begin{document}
	
	\maketitle
	
	\tableofcontents
	
	\clearpage
	
	\section*{Wstęp}
	\addcontentsline{toc}{section}{Wstęp}
	
	Celem ćwiczenia była analiza usług katalogowych eDirectory, dostępnych za pośrednictwem oprogramowania klienckiego sieci OES, zainstalowanego na stacjach roboczych.
	
	\section{Badanie mechanizmu mapowania zasobów usług katalogowych}
	Aby uzyskać dostęp do narzędzi i właściwości OES należy kliknąć prawym przyciskiem myszy na \textbf{ikonę OES Client} w pasku zadań. Następnie w celu mapowania dysku należy wybrać opcję \textbf{Mapowanie dysku sieciowego OES...}
	
	\framedfigure{images/1/1.1.png}{8cm}{Menu kontekstowe OES Client}{1.1}
	
	Istnieją dwie możliwości wprowadzenia ścieżki sieciowej do zasobu: użycie adresu serwera (notacja UNC) lub adresu odpowiedniego obiektu bazy danych usług katalogowych (drzewa).
	
	\clearpage
	
	\subsection{Adres serwera (notacja UNC)}
	
	\framedfigure{images/1/1.5.png}{8cm}{Mapowanie dysku używając adresu serwera o18}{1.2}
	
	\framedfigure{images/1/1.3.png}{8cm}{Przeglądarka zasobów}{1.3}
	
	\framedfigure{images/1/1.4.png}{14cm}{Zamapowany dysk w Eksploratorze plików}{1.4}
	
	\subsection{Adres obiektu bazy danych usług katalogowych}
	
	\framedfigure{images/1/1.7.png}{10cm}{Mapowanie dysku używając adresu obiektu bazy danych usług katalogowych}{1.5}
	
	\section{Identyfikacja zasobów sieciowych przyłączanych przez skrypt logowania kontenera}
	
	Przyłączone katalogi sieciowe są dostępne w Eksploratorze plików.
	
	\framedfigure{images/2/2.1.png}{12cm}{Przyłączone katalogi sieciowe}{2.1}
	
	Aby odnaleźć skrypt logowania należy w Eksploratorze Plików przejść do \textbf{Sieć>ZET\_TREE>prawym przyciskiem myszy na zet>Edytuj skrypt logowania kontenera eDir...}
	
	\framedfigure{images/2/2.2.png}{14cm}{Edytuj skrypt logowania}{2.2}
	
	Na poniższym zrzucie ekranu zaznaczono fragment skryptu, który realizuje przyłączanie katalogów sieciowych.
	
	\framedfigure{images/2/edited/2.3.png}{8cm}{Skrypt logowania}{2.3}
	
	\section{Badanie mechanizmu kopiowania NetWare}
	
	Mechanizm kopiowania NetWare jest dostępny przez ikonę \textbf{OES Client>Programy narzędziowe OES>Kopiowanie OES...}
	
	\framedfigure{images/3/3.1.png}{8cm}{Kopiowanie OES}{3.1}
	
	W nowym oknie należy wybrać plik lub folder do kopiowania.
	
	\framedfigure{images/3/3.2.png}{8cm}{Wybór pliku do kopiowania}{3.2}
	
	Następnie należy wpisać ścieżkę docelową i kliknąć OK.
	
	\framedfigure{images/3/3.3.png}{8cm}{Narzędzie do kopiowania plików OES}{3.3}
	
	Na poniższym zrzucie ekranu udokumentowano efekt użycia narzędzia kopiowania.
	
	\framedfigure{images/3/3.4.png}{12cm}{Efekt w Eksploratorze Plików}{3.4}
	
	\section{Badanie systemu komunikatów}
	
	W celu wysłania wiadomości należy kliknąć na ikonę klienta OES i przejść do \textbf{Programy narzędziowe OES>Wyślij wiadomość>Wyślij wiadomość do użytkowników...}
	
	\framedfigure{images/4/4.1.png}{6cm}{Okno kontekstowe}{4.1}
	
	Następnie należy wybrać serwer.
	
	\framedfigure{images/4/4.2.png}{8cm}{Wybór serwera}{4.2}
	
	Wybrany zostaje użytkownik lub grupa, do której zostanie wysłana wiadomość oraz zostaje wprowadzona treść wiadomości.
	
	\framedfigure{images/4/4.6.png}{8cm}{Wysyłanie wiadomości}{4.3}
	
	W nowym oknie można zobaczyć wyniki wysłania wiadomości.
	
	\framedfigure{images/4/4.5.png}{8cm}{Wyniki wysyłania wiadomości}{4.4}
	
	Poniżej załączono okno, które zostaje wyświetlone gdy otrzyma się wiadomość.
	
	\framedfigure{images/4/4.7.png}{8cm}{Otrzymanie komunikatu}{4.5}
	
	\section{Badanie i przedstawienie rekordów bazy danych usług katalogowych}

	Należy otworzyć menu kontekstowe OES Client i kliknąć \textbf{Zarządzanie użytkownikami dla ZET\_TREE}
	
	\framedfigure{images/5/5.1.png}{10cm}{Menu kontekstowe}{5.1}
	
	Poniżej zostaną wyświetlone wszystkie dostępne informacje.
	
	\begin{figure}[H]
		\centering
		\framedsubfigure{images/5/5.2.png}{7cm}{}{5.2}
		\framedsubfigure{images/5/5.3.png}{7cm}{}{5.3}
		\framedsubfigure{images/5/5.4.png}{7cm}{}{5.4}
		\framedsubfigure{images/5/5.5.png}{7cm}{}{5.5}
		\caption{Rekordy bazy danych usług katalogowych}
	\end{figure}
	
	\begin{figure}[H]
		\centering
		\framedsubfigure{images/5/5.6.png}{7cm}{}{5.6}
		\framedsubfigure{images/5/5.7.png}{7cm}{}{5.7}
		\framedsubfigure{images/5/5.8.png}{7cm}{}{5.8}
		\framedsubfigure{images/5/5.9.png}{7cm}{}{5.9}
		\caption{Rekordy bazy danych usług katalogowych}
	\end{figure}
	
	\begin{figure}[H]
		\centering
		\framedsubfigure{images/5/5.10.png}{5cm}{}{5.10}
			\caption{Rekordy bazy danych usług katalogowych}
	\end{figure}
	
	\section{Badanie zasobów dyskowych}
	
	Lista praw jest dostępna poprzez narzędzie Eksplorator Plików. Należy nacisnąć prawym przyciskiem myszy na dysk lub katalog i wybrać Prawa dysponenta...
	
	\framedfigure{images/6/6.1.png}{12cm}{Eksplorator plików}{6.1}
	
	\clearpage
	
	\subsection{Zasoby, do których użytkownik ma nadane bezpośrednie prawa}
	
	\subsubsection{Dysk F: (USR1:\textbackslash{}STUD\textbackslash{}337338)}
	
	\begin{figure}[H]
		\centering
		\framedsubfigure{images/6/6.2.png}{6cm}{Prawa dysponenta}{6.2}
		\framedsubfigure{images/6/6.3.png}{12cm}{Prawa dziedziczone}{6.3}
	\end{figure}
	
	\clearpage
	
	\subsection{Zasoby, do których użytkownik ma prawa wynikające z dziedziczenia}
	
	\subsubsection{Dysk Z: (NW5SYS\_PUB:\textbackslash{}PUBLIC)}
	
	Prawa wynikające z przynależności do grupy \textbf{EVERYONE.zet}.
	
	\begin{figure}[H]
		\centering
		\framedsubfigure{images/6/6.4.png}{6cm}{Prawa dysponenta}{6.4}
		\framedsubfigure{images/6/6.5.png}{12cm}{Prawa dziedziczone}{6.5}
	\end{figure}
	
	\subsubsection{Dysk U: (USR1:\textbackslash{}STUD)}
	
	Prawa wynikające z przynależności do grupy \textbf{STUD.zet}.
	
	\begin{figure}[H]
		\centering
		\framedsubfigure{images/6/6.6.png}{7cm}{Prawa dysponenta}{6.6}
		\framedsubfigure{images/6/6.7.png}{13cm}{Prawa dziedziczone}{6.7}
	\end{figure}
	
	\clearpage
	
	\subsubsection{Dysk X: (TEMP)}
	
	\textbf{Katalog TMP}
	
	Prawa wynikające z przynależności do grupy \textbf{STUD.zet}.
	
	\begin{figure}[H]
		\centering
		\framedsubfigure{images/6/6.8.png}{7cm}{Prawa dysponenta}{6.8}
		\framedsubfigure{images/6/6.9.png}{13cm}{Prawa dziedziczone}{6.9}
	\end{figure}
	
	\clearpage
	
	\textbf{Katalog TEMP}
	
	Prawa wynikające z przynależności do grup \textbf{EVERYONE.zet} i \textbf{STUD.zet}.
	
	\begin{figure}[H]
		\centering
		\framedsubfigure{images/6/6.10.png}{7cm}{Prawa dysponenta}{6.10}
		\framedsubfigure{images/6/6.11.png}{13cm}{Prawa dziedziczone}{6.11}
	\end{figure}
	
	\clearpage
	
	\textbf{Katalog WZORCE}
	
	Prawa wynikające z przynależności do grupy \textbf{STUD.zet}.
	
	\begin{figure}[H]
		\centering
		\framedsubfigure{images/6/6.12.png}{7cm}{Prawa dysponenta}{6.12}
		\framedsubfigure{images/6/6.13.png}{13cm}{Prawa dziedziczone}{6.13}
	\end{figure}
	
	\clearpage
	
	\subsubsection{Dysk Y: (FTP)}
	
	\textbf{Katalog FTP}
	
	Prawa wynikające z przynależności do grupy \textbf{EVERYONE.zet}.
	
	\begin{figure}[H]
		\centering
		\framedsubfigure{images/6/6.14.png}{7cm}{Prawa dysponenta}{6.14}
		\framedsubfigure{images/6/6.15.png}{13cm}{Prawa dziedziczone}{6.15}
	\end{figure}
	
	\section{Nadanie wybranych praw innemu użytkownikowi}
	
	Należy w \textbf{Eksploratorze plików} kliknąć prawym przyciskiem na katalog i wybrać \textbf{Prawa dysponenta...}
	
	\framedfigure{images/7/7.1.png}{15cm}{Eksplorator plików}{7.1}
	
	\clearpage
	
	Następnie wybrać użytkownika i kliknąć \textbf{Dodaj}.
	
	\framedfigure{images/7/7.2.png}{8cm}{Dodanie użytkownika}{7.2}
	
	Na koniec można ustawić uprawnienia dla użytkownika.
	
	\framedfigure{images/7/7.3.png}{8cm}{Zmiana uprawnień}{7.3}
	
	\section{Badanie wybranego obiektu}
	
	\subsection{Serwer}
	
	Należy kliknąć w ikonę klienta OES w pasku zadań i wybrać \textbf{Programy narzędziowe OES>Właściwości obiektu...}
	
	\framedfigure{images/8/8.1.png}{8cm}{Menu kontekstowe}{8.1}
	
	Należy wybrać obiekt, w tym przypadku serwer O18.
	
	\framedfigure{images/8/8.2.png}{8cm}{Wybór obiektu}{8.2}
	
	\framedfigure{images/8/8.3.png}{6cm}{Informacje o serwerze O18}{8.3}
	
	\subsection{Wolumen sieciowy}
	
	Należy w Eksploratorze plików nacisnąć prawym przyciskiem myszy na wolumen wybrać Właściwości i przejść do zakładek Statystyka wolumenu OES i Informacje o systemie OES.

	\begin{figure}[H]
		\centering
		\framedsubfigure{images/8/8.6.png}{7cm}{Statystyki wolumenu OES}{8.4}
		\framedsubfigure{images/8/8.10.png}{8cm}{Informacje o wolumenie}{8.5}
	\end{figure}
	
	\subsection{Katalog}
	
	Ponownie należy skorzystać z Eksploratora plików i zastosować te same kroki co dla wolumenu.
	
	\framedfigure{images/8/8.9.png}{7cm}{Informacje o katalogu}{8.6}
	
	\subsection{Plik}
	
	Analogicznie jak dla poprzednich obiektów.
	
	\framedfigure{images/8/8.8.png}{7cm}{Informacje o pliku}{8.7}
	
	\section{Badanie mechanizmu odzyskiwania usuniętych plików}
	
	W Eksploratorze plików należy wybrać dysk i kliknąć \textbf{Odzyskaj pliki...}
	
	\framedfigure{images/9/9.1.png}{14cm}{Eksplorator plików}{9.1}
	
	W nowym oknie można filtrować pliki i odczytać ich parametry, a także po wybraniu plików nacisnąć Odzyskaj plik/Odzyskaj wszystkie, aby je przywrócić.
	
	\framedfigure{images/9/9.2.png}{10cm}{Odzyskaj pliki sieciowe}{9.2}
	
	Poniższa lista prezentuje dostępne parametry dla plików:
	
	\begin{itemize}
		\item Nazwa pliku
		\item Data usunięcia
		\item Godzina usunięcia
		\item Rozmiar pliku
		\item Nazwa usuwającego
		\item Data ostatniej aktualizacji
		\item Godzina ostatniej aktualizacji
		\item Data utworzenia
		\item Godzina utworzenia
		\item Nazwa właściciela
		\item Data archiwizacji
		\item Godzina archiwizacji
		\item Nazwa archiwizatora
		\item Data ostatniego użycia 
	\end{itemize}
	
	\section{Modyfikacja skryptu logowania}
	
	W celu zmodyfikowania skryptu logowania należy kliknąć na ikonę klienta OES w pasku zadań i przejść do \textbf{Zarządzanie użytkownikami dla ZET\_TREE > Edytuj skrypt logowania...}
	
	\framedfigure{images/10/10.1.png}{12cm}{Menu kontekstowe}{10.1}
	
	Następnie należy wprowadzić zmiany i kliknąć OK.
	
	\framedfigure{images/10/10.2.png}{8cm}{Edytuj skrypt logowania}{10.2}
	
	Po ponownym zalogowaniu można sprawdzić efekt modyfikacji. W wierszu poleceń należy użyć komendy \textbf{set} w celu sprawdzenia zmiennych środowiskowych.
	
	\framedfigure{images/10/10.3.png}{10cm}{Efekt modyfikacji skryptu widoczny w CMD}{10.3}
	
	W Eksploratorze plików można znaleźć zamontowany katalog Ćwiczenie 2 jako dysk G.
	
	\framedfigure{images/10/10.4.png}{14cm}{Zamontowany katalog w Eksploratorze plików}{10.4}
	
	\section{Właściwości i parametry domyślnego profilu logowania}
	
	Aby znaleźć te dane należy w menu kontekstowym OES nacisnąć \textbf{Zarządzanie użytkownikami dla ZET\_TREE > Administrowanie profilami logowania...}
	
	\framedfigure{images/11/11.1.png}{10cm}{Menu kontekstowe}{11.1}
	
	Należy wprowadzić identyfikator użytkownika i kliknąć \textbf{Własności}.
	
	\framedfigure{images/11/11.2.png}{6cm}{Wprowadzanie identyfikator użytkownika}{11.2}
	
	Poniżej załączone są wszystkie dostępne właściwości i parametry domyślnego profilu logowania.
	
	\begin{figure}[H]
		\centering
		\framedsubfigure{images/11/11.3.png}{6cm}{Zakładka Poświadczenia}{11.3}
		\framedsubfigure{images/11/11.4.png}{6cm}{Zakładka eDirectory}{11.4}
		\framedsubfigure{images/11/11.5.png}{6cm}{Zakładka Skrypt}{11.5}
		\framedsubfigure{images/11/11.6.png}{6cm}{Zakładka Windows}{11.6}
		\framedsubfigure{images/11/11.7.png}{6cm}{Zakładka Usługi NMAS}{11.7}
		\framedsubfigure{images/11/11.8.png}{6cm}{Zakładka 802.1X}{11.8}
		\caption{Właściwości i parametry domyślnego profilu logowania}
	\end{figure}
	\clearpage
		
	\section{Lista drukarek}
	
	\subsection{Usługa OpenText iPrint}
	
	Listę drukarek można wyświetlić w przeglądarce internetowej wpisując adres: http://o18.iem.pw.edu.pl/ipp.
	
	\framedfigure{images/12/12.1.png}{16cm}{Lista drukarek dostępnych w ramach usługi OpenText iPrint}{12.1}
	
	\subsection{Aplet Urządzenia i drukarki w systemie Windows}
	
	Aby znaleźć ten aplet należy otworzyć \textbf{Panel Sterowania} i przejść do \textbf{Sprzęt i dźwięk > Urządzenia i drukarki}.
	
	\framedfigure{images/12/12.2.png}{10cm}{Lista drukarek w aplecie Urządzenia i drukarki}{12.2}
	
	\section{Badanie parametrów usługi drukowania sieciowego OpenText iPrint i własności drukarek}
	
	Aby otworzyć ustawienia należy w pasku zadań kliknąć na ikonę iPrint i nacisnąć \textbf{Ustawienia iPrint}.
	
	\framedfigure{images/13/13.1.png}{6cm}{Ustawienia iPrint}{13.1}
	
	Poniżej zaprezentowano wszystkie zakładki okna ustawień.
	
	\begin{figure}[H]
		\centering
		\framedsubfigure{images/13/13.2.png}{8cm}{O programie}{13.2}
		\framedsubfigure{images/13/13.3.png}{8cm}{Proxy}{13.3}
		\framedsubfigure{images/13/13.4.png}{8cm}{Hasła}{13.4}
		\caption{Zakładki okna ustawień iPrint}
	\end{figure}
	
	\begin{figure}[H]
		\centering
		\framedsubfigure{images/13/13.5.png}{7cm}{Ikona podajnika}{13.5}
		\framedsubfigure{images/13/13.6.png}{7cm}{Potwierdzenie}{13.6}
		\caption{Zakładki okna ustawień iPrint}
	\end{figure}
	
	Właściwości drukarki są dostępne dzięki narzędziu \textbf{Urządzenia i drukarki}, wspomnianym w poprzednim zadaniu.
	
	\framedfigure{images/13/13.7.png}{12cm}{Urządzenia i drukarki}{13.7}
	
	\begin{figure}[H]
		\centering
		\framedsubfigure{images/13/13.18.png}{6cm}{Ogólne}{13.8}
		\framedsubfigure{images/13/13.19.png}{6cm}{Udostępnianie}{13.9}
		\framedsubfigure{images/13/13.20.png}{6cm}{Porty}{13.10}
		\framedsubfigure{images/13/13.21.png}{6cm}{Zaawansowane}{13.11}
		\framedsubfigure{images/13/13.22.png}{6cm}{Zarządzanie kolorami}{13.12}
		\caption{Właściwości drukarki}
	\end{figure}
	
	\begin{figure}[H]
		\centering
		
		\framedsubfigure{images/13/13.23.png}{6cm}{Zabezpieczenia}{13.13}
		\framedsubfigure{images/13/13.24.png}{6cm}{Ustawienia urządzenia}{13.14}
		\caption{Właściwości drukarki}
	\end{figure}
	
	Informacje o drukarce można też znaleźć używając strony w przeglądarce internetowej, zaprezentowanej w poprzednim zadaniu.
	
	\framedfigure{images/13/13.17.png}{12cm}{Właściwości drukarki w Micro Focus iPrint}{13.15}
	
	\section{Pomiar czasu logowania w zależności od użytego adresu}
	
	Aby dokonać pomiaru czasu logowania należy wylogować się, zmienić serwer do którego się logujemy, kliknąć Zaloguj, zmierzyć czas, a na koniec wyświetlić okno Połączenia OES.
	
	\subsection{Nazwy DNS}
	
	\begin{figure}[H]
		\centering
		\framedsubfigure{images/14/14.1.jpg}{8cm}{}{14.1}
		\framedsubfigure{images/14/14.2.png}{6cm}{}{14.2}
		\caption{Nazwa DNS: o17.iem.pw.edu.pl; Czas: 13,410 s}
	\end{figure}
	
	\begin{figure}[H]
		\centering
		\framedsubfigure{images/14/14.3.jpg}{8cm}{}{14.3}
		\framedsubfigure{images/14/14.4.png}{6cm}{}{14.4}
		\caption{Nazwa DNS: o18.iem.pw.edu.pl; Czas: 9,98 s}
	\end{figure}
	
	\begin{figure}[H]
		\centering
		\framedsubfigure{images/14/14.5.jpg}{8cm}{}{14.5}
		\framedsubfigure{images/14/14.6.png}{6cm}{}{14.6}
		\caption{Nazwa DNS: o19.iem.pw.edu.pl; Czas: 13,28 s}
	\end{figure}
	
	\subsection{Adresy IP}
	
	\begin{figure}[H]
		\centering
		\framedsubfigure{images/14/14.7.jpg}{6cm}{}{14.7}
		\framedsubfigure{images/14/14.12.png}{6cm}{}{14.8}
		\caption{Adres: 10.42.1.24; Czas: 12,95 s}
	\end{figure}
	
	\begin{figure}[H]
		\centering
		\framedsubfigure{images/14/14.20.jpg}{6cm}{}{14.9}
		\framedsubfigure{images/14/14.8.png}{6cm}{}{14.10}
		\caption{Adres: 10.42.1.25; Czas: 10,45 s}
	\end{figure}
	
	\begin{figure}[H]
		\centering
		\framedsubfigure{images/14/14.9.jpg}{6cm}{}{14.11}
		\framedsubfigure{images/14/14.10.png}{6cm}{}{14.12}
		\caption{Adres: 10.42.1.26; Czas: 12,21 s}
	\end{figure}
	
	\subsection{Nazwy symboliczne}
	
	\begin{figure}[H]
		\centering
		\framedsubfigure{images/14/14.11.jpg}{6cm}{}{14.13}
		\framedsubfigure{images/14/14.12.png}{6cm}{}{14.14}
		\caption{Nazwa symboliczna: O17 ; Czas: 14,01 s}
	\end{figure}
	
	\begin{figure}[H]
		\centering
		\framedsubfigure{images/14/14.13.jpg}{6cm}{}{14.15}
		\framedsubfigure{images/14/14.18.png}{6cm}{}{14.16}
		\caption{Nazwa symboliczna: O18; Czas: 13,07 s}
	\end{figure}
	
	\begin{figure}[H]
		\centering
		\framedsubfigure{images/14/14.15.jpg}{6cm}{}{14.17}
		\framedsubfigure{images/14/14.14.png}{6cm}{}{14.18}
		\caption{Nazwa symboliczna: O19; Czas: 12,85 s}
	\end{figure}
	
	\subsection{Błędne nazwy}
	
	\framedfigure{images/14/14.17.jpg}{8cm}{Błędna nazwa; Czas: 47,31 s}{14.19}
	
	\framedfigure{images/14/14.19.jpg}{8cm}{Brak nazwy; Czas: 0 s}{14.20}
	
	\section{Własne wrażenia z użytkowania usług katalogowych eDirectory}
	
	Uważam, że usługi katalogowe eDirectory, badane przy użyciu klienta OES, są rozwiązaniem spójnym i funkcjonalnym. Usługa oferuje głęboką integrację z systemem Windows, dzięki czemu jest to bardzo wygodne narzędzie.
	
	Pozytywnie odbieram fakt, że kluczowe funkcje, takie jak badanie praw dysponenta, odzyskiwanie plików czy modyfikacja skryptów logowania, są dostępne bezpośrednio z poziomu klienta i Eksploratora plików.

	Podsumowując, uważam klienta OES za potężne narzędzie. Mimo iż interfejs niektórych narzędzi wydaje się przestarzały, to kluczowa funkcjonalność zarządzania zasobami i prawami stoi na wysokim, profesjonalnym poziomie.
	
\end{document}