% =============================================================================
% DOCUMENT SETUP & LANGUAGE
% =============================================================================

\usepackage{polyglossia}
\setmainlanguage{polish}

% =============================================================================
% FONT
% =============================================================================
\usepackage{fontspec}
\newfontfamily{\nerdfont}{Arimo Nerd Font}
\setsansfont{Arimo Nerd Font}
\setmonofont{Iosevka Nerd Font}
\renewcommand{\familydefault}{\sfdefault}

% =============================================================================
% COLORS
% =============================================================================
\usepackage{xcolor} % Driver for color support
\definecolor{accent}{HTML}{8839ef} % Main accent color
\definecolor{codegray}{rgb}{0.95,0.95,0.95} % Background for code blocks
\definecolor{commentgreen}{rgb}{0,0.4,0}    % Color for comments in code
\definecolor{keywordblue}{rgb}{0,0,0.6}     % Color for keywords in code
\definecolor{sysop}{HTML}{ed7c50}
\definecolor{fizyka}{HTML}{8839ef}
\definecolor{jimp}{HTML}{dd7878}
\definecolor{podstawyinformatyki}{HTML}{1e66f5}

% =============================================================================
% PAGE LAYOUT
% =============================================================================
\usepackage[a4paper, margin=2.5cm]{geometry} % Page dimensions
\usepackage{titlesec} % For customizing section titles

% Adjust spacing for section titles
\titlespacing*{\section}{0pt}{1ex}{1ex}
\titlespacing*{\subsection}{0pt}{0.5ex}{0.5ex}

% =============================================================================
% GRAPHICS, FIGURES & BOXES
% =============================================================================
\usepackage{graphicx}   % For including images
\usepackage{float}      % Improved float control (e.g., [H] placement)
\usepackage[most]{tcolorbox} % For creating colored boxes
\usepackage{subcaption} % For subfigures

% =============================================================================
% TABLES
% =============================================================================
\usepackage{tabularx}   % For tables with adjustable-width columns
\usepackage{colortbl}   % For colored table cells
\usepackage{booktabs}   % For high-quality table rules
\usepackage{array}      % Enhanced array and tabular environments

% Adjust table row height and column separation
\renewcommand{\arraystretch}{1.25} 
\setlength{\tabcolsep}{10pt}

% =============================================================================
% CODE LISTINGS
% =============================================================================
\usepackage{listings} % For typesetting code

\lstdefinestyle{mystyle}{
	language=c,
	commentstyle=\color{commentgreen},
	keywordstyle=\color{accent},
	numberstyle=\tiny\color{gray},
	stringstyle=\color{purple},
	basicstyle=\ttfamily\footnotesize,
	breakatwhitespace=false,
	breaklines=true,
	captionpos=b,
	keepspaces=true,
	numbers=left,
	numbersep=5pt,
	showspaces=false,
	showstringspaces=false,
	showtabs=false,
	tabsize=4
}

% 2. Zdefiniuj nowy styl pudełka dla kodu
\newtcblisting{mylisting}{
	arc=2mm,
	top=0mm,
	bottom=0mm,
	left=5mm,
	right=0mm,
	boxrule=0.5pt,
	colback=codegray,
	colframe=accent,
	listing only,
	listing options={style=mystyle},
}
% =============================================================================
% BIBLIOGRAPHY & HYPERLINKS
% =============================================================================
\usepackage[backend=biber]{biblatex}
\addbibresource{bibliography.bib} % Bibliography database file

\usepackage{hyperref} % For hyperlinks and PDF metadata
\usepackage{xurl}     % For handling long URLs

\hypersetup{
	colorlinks=true,
	allcolors=black % Set all link colors to black
}

% =============================================================================
% CUSTOM COMMANDS & ENVIRONMENTS
% =============================================================================

% --- Custom Section Boxes ---
% Uses tcolorbox to create styled boxes for section titles.
\newcommand{\mysectionbox}[1]{%
	\begin{tcolorbox}[
		enhanced, colback=accent, colframe=accent, boxrule=1pt, arc=2mm
		]
		\color{white}\thesection\hspace{1em}#1
	\end{tcolorbox}%
}

\newcommand{\mysubsectionbox}[1]{%
	\begin{tcolorbox}[
		enhanced, colback=accent!60, colframe=accent!60, arc=2mm, boxrule=1pt
		]
		\color{black}\thesubsection\hspace{1em}#1
	\end{tcolorbox}%
}

\newcommand{\mysubsubsectionbox}[1]{%
	\begin{tcolorbox}[
		enhanced, colback=accent!20, colframe=accent!20, arc=2mm, boxrule=1pt
		]
		\color{black}\thesubsubsection\hspace{1em}#1
	\end{tcolorbox}%
}

% Box for unnumbered sections (e.g., Bibliography)
\newcommand{\mysectionboxstar}[1]{
	\begin{tcolorbox}[
		enhanced, colback=accent, colframe=accent, boxrule=1pt, arc=2mm
		]
		\color{white}#1
	\end{tcolorbox}%
}

% --- Framed Figures ---
% Creates a figure with a colored frame and shadow.
\newcommand{\framedfigure}[4]{%
	\begin{figure}[H]
		\centering
		\begin{tcolorbox}[
			enhanced, colframe=accent, colback=white, boxrule=1.5pt,
			sharp corners, boxsep=0pt, left=0pt, right=0pt, top=0pt, bottom=0pt,
			hbox, drop shadow
			]
			\includegraphics[width=#2]{#1}
		\end{tcolorbox}
		\caption{#3}
		\label{#4}
	\end{figure}
}

% Creates a subfigure with a colored frame and shadow.
\newcommand{\framedsubfigure}[4]{%
	\begin{subfigure}{#2}
		\centering
		\begin{tcolorbox}[
			enhanced, colframe=accent, colback=white, boxrule=1.5pt,
			sharp corners, boxsep=0pt, left=0pt, right=0pt, top=0pt, bottom=0pt,
			hbox, drop shadow
			]
			\includegraphics[width=\linewidth]{#1}
		\end{tcolorbox}
		\caption{#3}
		\label{#4}
	\end{subfigure}
}

% =============================================================================
% TITLE FORMATTING
% =============================================================================
% Apply the custom boxes to section titles.
\titleformat{\section}{\normalfont}{}{0pt}{\mysectionbox}[]
\titleformat{\subsection}{\normalfont}{}{0pt}{\mysubsectionbox}[]
\titleformat{\subsubsection}{\normalfont}{}{0pt}{\mysubsubsectionbox}[]
\titleformat{name=\section, numberless}{\normalfont}{}{0pt}{\mysectionboxstar}[]

% =============================================================================
% OTHER SETTINGS
% =============================================================================
\let\oldtextbf\textbf
\renewcommand{\textbf}[1]{%
	\textcolor{accent}{\oldtextbf{#1}}%
}

% Author icon
\newcommand{\authoricon}{{\nerdfont\textcolor{accent}{ \space}}}

% Date icon
\newcommand{\dateicon}{{\nerdfont\textcolor{accent}{ \space}}}