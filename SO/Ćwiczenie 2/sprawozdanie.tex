\documentclass[]{article}

\usepackage[polish]{babel}
\usepackage[T1]{fontenc}
\usepackage{graphicx}
\usepackage{float}
\usepackage{helvet}
\usepackage[most]{tcolorbox}
\usepackage{titlesec}
\usepackage{xcolor}
\usepackage[a4paper, margin=2.5cm]{geometry}
\usepackage[colorlinks=true, allcolors=black]{hyperref}
\usepackage{listings}
\usepackage{tabularx}
\usepackage{colortbl}   % Do kolorowania wierszy
\usepackage{booktabs}   % Dla estetycznych linii
\usepackage{subcaption}
\usepackage{array}
\usepackage[backend=biber]{biblatex}
\addbibresource{bibliography.bib}
\usepackage{hyperref}
\usepackage{xurl}
\usepackage{titlesec}
\usepackage{amsmath}

% Zmniejszenie marginesów dla \section
\titlespacing*{\section}
{0pt} % Odstęp po lewej (margines)
{1ex} % Odstęp PRZED tytułem (zmniejszony)
{1ex} % Odstęp PO tytule (zmniejszony)

% Zmniejszenie marginesów dla \subsection
\titlespacing*{\subsection}
{0pt} % Odstęp po lewej
{0.5ex} % Odstęp PRZED tytułem
{0.5ex} % Odstęp PO tytule

% Font family
\renewcommand{\familydefault}{\sfdefault}

%opening

\definecolor{accent}{HTML}{8839ef}

% 1. Define a new command that takes one argument (#1) for the title text
\newcommand{\mysectionbox}[1]{%
	\begin{tcolorbox}[
		enhanced,
		colback=accent!100!accent,
		colframe=accent!100!accent,
		sharp corners,
		boxrule=1pt,
		]
		\color{white}\thesection\hspace{1em}#1
	\end{tcolorbox}%
}

% 2. Use the simple new command inside \titleformat
\titleformat{\section}
{\normalfont}
{}
{0pt}
{\mysectionbox} % Use the new command here (without arguments)
[]

\newcommand{\mysubsectionbox}[1]{%
	\begin{tcolorbox}[
		enhanced,
		colback=accent!60, % A lighter (20%) tint for the background
		colframe=accent!60,    % Keep the original frame color
		sharp corners,
		boxrule=1pt,
		]
		% Use a dark color for text on a light background
		\color{black}\thesubsection\hspace{1em}#1
	\end{tcolorbox}%
}

\titleformat{\subsection}
{\normalfont} % Set font style for subsections
{}
{0pt}
{\mysubsectionbox}
[]

\newcommand{\framedfigure}[4]{%
	% #1: image file
	% #2: width
	% #3: caption text
	% #4: label for referencing
	\begin{figure}[H]
		\centering
		\begin{tcolorbox}[
			enhanced,
			colframe=accent,
			colback=white,
			boxrule=1.5pt,
			sharp corners,
			boxsep=0pt,
			left=0pt,
			right=0pt,
			top=0pt,
			bottom=0pt,
			hbox,	
			drop shadow
			]
			\includegraphics[width=#2]{#1}
		\end{tcolorbox}
		\caption{#3}
		\label{#4}
	\end{figure}
}

\newcommand{\framedsubfigure}[4]{%
	% #1: image file
	% #2: width
	% #3: caption text
	% #4: label for referencing
	\begin{subfigure}{#2}
		\centering
		\begin{tcolorbox}[
			enhanced,
			colframe=accent,
			colback=white,
			boxrule=1.5pt,
			sharp corners,
			boxsep=0pt,
			left=0pt,
			right=0pt,
			top=0pt,
			bottom=0pt,
			hbox,	
			drop shadow
			]
			\includegraphics[width=\linewidth]{#1}
		\end{tcolorbox}
		\caption{#3}
		\label{#4}
	\end{subfigure}
}

% 2. Save the original \textbf command
\let\oldtextbf\textbf
\renewcommand{\textbf}[1]{%
	\textcolor{accent}{\oldtextbf{#1}}%
}

\definecolor{codegray}{rgb}{0.95,0.95,0.95}
\definecolor{commentgreen}{rgb}{0,0.4,0}
\definecolor{keywordblue}{rgb}{0,0,0.6}

\lstset{
	backgroundcolor=\color{codegray}, % Background color
	rulecolor=\color{accent},
	commentstyle=\color{commentgreen}, % Comment color
	keywordstyle=\color{accent},   % Keyword color
	numberstyle=\tiny\color{gray},    % Line number style
	stringstyle=\color{purple},       % String color
	basicstyle=\ttfamily\footnotesize,  % Font style for code
	breakatwhitespace=false,          % Don't break lines only at whitespace
	breaklines=true,                  % Break long lines automatically
	captionpos=b,                     % Caption position (b=bottom, t=top)
	frame=single,                     % Adds a frame around the code
	keepspaces=true,                  % Keep spaces in the code
	numbers=left,                     % Line numbers on the left
	numbersep=5pt,                    % Distance of numbers from the code
	showspaces=false,                 % Don't show spaces as special characters
	showstringspaces=false,           % Don't show spaces in strings
	showtabs=false,                   % Don't show tabs as special characters
	tabsize=2                         % Tab size
}

\newenvironment{conditions}
{\par\vspace{\abovedisplayskip}\noindent\begin{tabular}{>{$}l<{$} @{${}={}$} l}}
	{\end{tabular}\par\vspace{\belowdisplayskip}}
	
% Zwiększamy odstępy pionowe (padding góra/dół)
\renewcommand{\arraystretch}{1.25} 
% Zwiększamy odstępy poziome (padding lewo/prawo)
\setlength{\tabcolsep}{10pt}

\begin{document}
	
	\tableofcontents
	
	\clearpage
	
	\section{Cel}
	Ćwiczenie podzielono na dwie zasadnicze części.
	
	\subsection{Część I: Wyznaczanie objętości walca}
	
	Celem pierwszej części jest wyznaczenie objętości walca. Zadanie polega na precyzyjnym zmierzeniu wymiarów geometrycznych obiektu (wysokości i średnicy) przy użyciu suwmiarki oraz śruby mikrometrycznej. Następnie, na podstawie uzyskanych pomiarów, należy obliczyć objętość walca wraz z oszacowaniem niepewności pomiarowej.
	
	\subsection{Część II: Weryfikacja prawa Ohma}
	
	Druga część ćwiczenia poświęcona jest eksperymentalnej weryfikacji prawa Ohma. W tym celu należy dokonać pomiarów natężenia prądu elektrycznego płynącego przez opornik oraz napięcia na jego zaciskach. Końcowym etapem jest analiza zebranych danych, obejmująca obliczenie niepewności pomiarowych oraz graficzne przedstawienie i zbadanie zależności między natężeniem a napięciem.
	
	\section{Wstęp teoretyczny}
	
	\subsection{Pomiar pośredni i bezpośredni}
	\label{pomiar}
	
	\textbf{Pomiar bezpośredni} ma miejsce wtedy, gdy wartość mierzonej wielkości fizycznej jest uzyskiwana bezpośrednio z odczytu na przyrządzie pomiarowym. Wynik jest natychmiast widoczny na skali lub wyświetlaczu urządzenia, bez konieczności wykonywania dodatkowych obliczeń. Przykładami takiego pomiaru są:
	
	\begin{itemize}
		\item Pomiar długości linijką lub suwmiarką.
		
		\item Pomiar masy przy użyciu wagi.
		
		\item Pomiar temperatury termometrem.
		
		\item Pomiar napięcia woltomierzem.
		
		\item Pomiar czasu stoperem.
	\end{itemize}
	
	\textbf{Pomiar pośredni} jest stosowany, gdy nie można zmierzyć szukanej wielkości wprost. Polega on na wyznaczeniu wartości tej wielkości na podstawie znanego związku matematycznego (wzoru fizycznego), który łączy ją z innymi wielkościami, które są mierzone bezpośrednio. Przykładami pomiaru pośredniego są:
	\begin{itemize}
		\item Wyznaczanie gęstości ($\rho$): Mierzy się bezpośrednio masę (m) i objętość (V), a następnie oblicza gęstość ze wzoru $\rho=\frac{m}{V}$.
		
		\item Wyznaczanie prędkości średniej (v): Mierzy się bezpośrednio przebytą drogę (s) i czas (t), a następnie oblicza prędkość ze wzoru $v=\frac{s}{t}$.
		
		\item Wyznaczanie oporu (R) z prawa Ohma: Mierzy się bezpośrednio napięcie (U) i natężenie (I), a następnie oblicza opór ze wzoru $R=\frac{U}{I}$.
		
		\item Wyznaczanie objętości walca (V): Mierzy się bezpośrednio wysokość (H) i średnicę (d), a następnie oblicza objętość ze wzoru $V = \frac{\pi d^2 H}{4}$.
	\end{itemize}
	
	\subsection{Niepewności pomiarowe typu A i B}
	
	\textbf{Niepewność pomiarowa} to parametr charakteryzujący rozrzut wartości, które można w uzasadniony sposób przypisać wielkości mierzonej. Niepewność pomiarową można obliczyć używając \textbf{metody typu A i B}.
	
	\subsubsection{Metoda typu A}
	
	\textbf{Metoda typu A} wynika bezpośrednio z analizy statystycznej serii powtarzalnych pomiarów, które traktuje się jako próbę statystyczną podlegającą rozkładowi prawdopodobieństwa. W celu obliczenia tej niepewności należy najpierw policzyć wartość średnią z pomiarów jako:
	
	\begin{equation}
		\bar{x} = \frac{1}{n}\sum_{i=1}^{n}x_i
		\label{eq:srednia}
	\end{equation}
	
	\noindent Korzystając z poniższego wzoru można obliczyć niepewność metodą typu A:
	
	\begin{equation}
		u(x) = \sqrt{\frac{1}{n(n-1)}\sum_{i=1}^{n}(x_i-\bar{x})^2}
		\label{eq:metoda_a}
	\end{equation}
	Gdzie:
	\begin{conditions}
		n & liczba pomiarów \\
		\bar{x} & średnia arytmetyczna pomiarów \\
	\end{conditions}
	
	\subsubsection{Metoda typu B}
	
	\textbf{Metoda typu B} wyznaczania niepewności opiera się na wszelkich dostępnych informacjach innych niż analiza statystyczna bieżącej serii pomiarów.Źródłem informacji może być osąd naukowy, ale także:
	
	\begin{itemize} 
		\item dane od producenta przyrządu (np. jego klasa dokładności), 
		\item informacje ze świadectwa wzorcowania, 
		\item dane z poprzednich pomiarów lub literatury. 
	\end{itemize}
	
	W celu obliczenia niepewności metodą typu B, należy policzyć niepewność wzorcowania $\Delta x$ oraz niepewność eksperymentatora $\Delta x_e$.
	
	\medspace
	
	\textbf{Niepewność wzorcowania $\Delta x$}
	Jest to niepewność związana z użytym narzędziem. Przeważnie jako maksymalną wartość przyjmujemy najmniejszą podziałkę przyrządu pomiarowego \cite{wyk_wstep}.
	
	\medspace
	
	\textbf{Niepewność eksperymentatora $\Delta x_e$}
	Jest to niepewność wprowadzana przez osobę wykonującą pomiar. Zazwyczaj przyjmuje się jako wartość minimalną $\Delta x_e = \frac{1}{2}\Delta x$ (wyjątkiem są pomiary, gdzie niepewność wzorcowania jest dużo mniejsza niż dokładność działania eksperymentatora).
	
	\medspace
	
	\noindent Niepewność pomiaru typu B można obliczyć korzystając z poniższych wzorów:
	
	\begin{equation}
	u_B(x) = \sqrt{\frac{\Delta x^2}{3}} \quad \text{lub} \quad u_B(x)=\sqrt{\frac{\Delta x^2}{3} + \frac{\Delta x_e^2}{3}}
	\label{metoda_b}
	\end{equation}
	
	\subsection{Niepewność całkowita}
	
	\textbf{Niepewność całkowita} to pojedyncza wartość, która reprezentuje łączny efekt wszystkich zidentyfikowanych źródeł niepewności wpływających na końcowy wynik pomiaru. Pozwala na połączenie ze sobą niepewności $u_A(x)$ $u_B(x)$
	
	Aby policzyć niepewność całkowitą należy skorzystać ze wzoru:
	
	\begin{equation}
		u(x) = \sqrt{u_A^2(x) + u_B^2(x)}
		\label{eq:niepewnosc_calkowita}
	\end{equation}
	
	\subsection{Propagacja niepewności}
	\textbf{Propagacja niepewności} to metoda matematyczna służąca do obliczenia, jaka jest niepewność ostatecznego wyniku, jeśli ten wynik jest obliczany ze wzoru na podstawie innych wielkości, które zostały zmierzone z własnymi niepewnościami.
	
	\subsubsection{Przykład}
	Wielkość y jest kombinacją wielkości a, b, c, które są mierzone bezpośrednio z całkowitymi (typu A i/lub typu B) niepewnościami odpowiednio $u_a$, $u_b$, $u_c$. Dla takiego przypadku niepewność złożona może być liczona z prawa propagacji niepewności.
	
	\begin{equation}
		y = f(a,b,c)
	\end{equation}
	
	\begin{equation}
		u(y) = \sqrt{\left(\frac{\partial f}{\partial a}\right)^2u_a^2 + \left(\frac{\partial f}{\partial b}\right)^2u_b^2 + \left(\frac{\partial f}{\partial c}\right)^2u_c^2}
		\label{eq:prawo_propagacji}
	\end{equation}
	
	\subsection{Niepewność rozszerzona}
	\textbf{Niepewność rozszerzona} $U(x)$ – wielkość określająca przedział wokół wyniku pomiaru, od którego oczekuje się, że obejmie przeważającą część
	wyników. \cite{wyk_wstep}
	
	\medspace
	
	\noindent \textbf{Współczynnik rozszerzenia k} to współczynnik liczbowy, stosowany do obliczenia niepewności rozszerzonej. W praktyce laboratoryjnej, zaleca się przyjęcie wartości $k = 2$.
	
	\medspace
	
	\noindent \textbf{Niepewność rozszerzoną} wyraża się jako:
	
	\begin{equation}
		U(x) = k * u(x)
	\end{equation}
	
	\subsection{Metoda najmniejszych kwadratów}
	
	\textbf{Metoda najmniejszych kwadratów (MNK)} to standardowa procedura analityczna służąca do znajdowania "najlepszego dopasowania" modelu matematycznego (np. linii prostej $y=ax+b$) do zbioru punktów danych.
	
	\medspace
	
	Polega ona na znalezieniu takich wartości parametrów $a$ i $b$, dla których suma kwadratów odchyleń (różnic między wartościami zmierzonymi $y_i$ a wartościami przewidzianymi przez model $ax_i+b$) jest najmniejsza.
	
	Matematycznie, sprowadza się to do minimalizacji następującej funkcji:
	
	\begin{equation}
		\sum_{i=1}^{n}[y_i - (ax_i + b)]^2 = \text{min}
	\end{equation}
	
	Gdzie:
	
	\begin{conditions}
		n & liczba pomiarów\\
		a & współczynnik kierunkowy prostej\\
		b & wyraz wolny prostej\\
	\end{conditions}
	
	\subsection{Prawo Ohma}
	\textbf{Prawo Ohma} stwierdza, że dla przewodnika utrzymywanego w stałej temperaturze, natężenie prądu elektrycznego ($I$), czyli uporządkowanego ruchu elektronów, jest wprost proporcjonalne do napięcia ($U$) przyłożonego do przewodnika.
	
	\medspace
	
	W rezultacie, stosunek $\frac{U}{I}$ jest dla danego przewodnika stały. Tę stałą wielkość, będącą miarą oporu stawianego przez przewodnik, nazywa się \textbf{oporem elektrycznym} ($R$).
	
	\begin{equation}
		R = \frac{U}{I}
	\end{equation}
	
	Jednostką oporu elektrycznego w układzie SI jest \textbf{om} oznaczany symbolem $\Omega$.
	
	\medspace
	
	Aby eksperymentalnie zweryfikować tę zależność, konieczne jest prawidłowe wpięcie przyrządów pomiarowych w obwód:
	
	\begin{itemize}
		\item Woltomierz (służący do pomiaru napięcia $U$) należy zawsze podłączać \textbf{równolegle} do elementu, na którym mierzymy napięcie (np. równolegle do opornika).
		
		\item Amperomierz (służący do pomiaru natężenia $I$) należy zawsze podłączać \textbf{szeregowo} w obwodzie, tak aby mierzony prąd musiał przez niego przepłynąć.
	\end{itemize}
	
	\subsection{Wyznaczanie objętości walca}
	
	\textbf{Wyznaczenie objętości walca} ($V$) jest pomiarem pośrednim (opisany w sekcji \ref{pomiar}). Wymaga on przeprowadzenia dwóch pomiarów bezpośrednich:
	
	\begin{itemize}
		\item $H$ = wysokość walca (mierzona za pomocą suwmiarki)
		\item $d$ = średnica podstawy (mierzona za pomocą śruby mikrometrycznej)
	\end{itemize}
	
	\noindent Następnie, wartość objętości obliczana jest na podstawie zależności matematycznej:
	
	\begin{equation} V = \frac{\pi d^2 H}{4} \end{equation}
	
	\noindent Ostateczna niepewność pomiaru pośredniego $u(V)$ musi zostać wyznaczona metodą propagacji niepewności, uwzględniając niepewności pomiarów bezpośrednich $u(H)$ i $u(d)$.
	
	\clearpage

	\section{Układy i przyrządy pomiarowe}
	
	\subsection{Część I: Wyznaczanie objętości walca}
	
	\begin{figure}[H]
		\centering
		\includegraphics[width=6cm]{images/schemat_walca.png}
		\caption{Schemat walca}
		\label{fig:walec}
	\end{figure}
	
	W skład układu pomiarowego wchodziły:
	\begin{itemize}
		\item Aluminiowy walec
		\item Suwmiarka noniuszowa analogowa - służąca do pomiaru wysokości walca (H)
		\item Śruba mikrometryczna - użyta do pomiaru średnicy walca (d)
	\end{itemize}
	
	\begin{figure}[H]
		\centering
		\begin{subfigure}{0.45\textwidth}
			\centering
			\includegraphics[width=6cm]{images/suwmiarka.jpg}
			\caption{Suwmiarka noniuszowa analogowa \cite{suwmiarka}}
			\label{fig:suwmiarka}
		\end{subfigure}
		\hfill
		\begin{subfigure}{0.45\textwidth}
			\centering
			\includegraphics[width=6cm]{images/sruba_mikrometryczna.jpg}
			\caption{Śruba mikrometryczna \cite{sruba}}
			\label{fig:sruba}
		\end{subfigure}
		\caption{Przyrządy użyte do mierzenia walca}
	\end{figure}
	
	W ramach eksperymentu dokonano 15 pomiarów wysokości walca suwmiarką noniuszową analogową o dokładności 0,1 mm oraz 15 pomiarów średnicy walca śrubą mikrometryczną o dokładności 0,02 mm.
	\clearpage
	
	\subsection{Część II: Weryfikacja prawa Ohma}
	Układ pomiarowy drugiej części laboratorium składał się z:
	\begin{itemize}
		\item Płytki stykowej (Rysunek \ref{fig:plytka}) - z wpiętym opornikiem R4
		\item Miernika uniwersalnego UM-112B (Rysunek \ref{fig:woltomierz}) - użytego jako woltomierz
		\item Cyfrowego miernika uniwersalnego M-3800 (Rysunek \ref{fig:amperomierz}) - użytego jako amperomierz
		\item Zasilacza laboratoryjnego DF1730SL5A (Rysunek \ref{fig:zasilacz})
		\item Przewodów
	\end{itemize}
	
	\begin{figure}[H]
		\centering
		\begin{subfigure}{0.45\textwidth}
			\centering
			\includegraphics[width=7cm]{images/schemat_ukladu_pomiarowego.png}
			\caption{Schemat podłączenia urządzeń}
			\label{fig:schemat_ukladu}
		\end{subfigure}
		\hfill
		\begin{subfigure}{0.45\textwidth}
			\centering
			\includegraphics[width=7cm]{images/plytka.jpg}
			\caption{Płytka stykowa}
			\label{fig:plytka}
		\end{subfigure}
		\caption{Układ pomiarowy}
	\end{figure}
	
	\begin{figure}[H]
		\centering
		\begin{subfigure}{0.45\textwidth}
			\centering
			\includegraphics[width=6cm]{images/woltomierz.jpg}
			\caption{Miernik uniwersalny UM-112B}
			\label{fig:woltomierz}
		\end{subfigure}
		\hfill
		\begin{subfigure}{0.45\textwidth}
			\centering
			\includegraphics[width=4cm]{images/amperomierz.jpg}
			\caption{Cyfrowy miernik uniwersalny M-3800}
			\label{fig:amperomierz}
		\end{subfigure}
		
		\caption{Mierniki}
	\end{figure}
	
	\begin{figure}[H]
		\centering
		\includegraphics[width=5.9cm]{images/zasilacz.jpg}
		\caption{Zasilacz laboratoryjny DF1730SL5A}
		\label{fig:zasilacz}
	\end{figure}

	W celu zmierzenia napięcia na oporniku R4 oraz natężenia płynącego przez niego prądu, woltomierz wpięto równolegle, a amperomierz szeregowo.
	
	W trakcie eksperymentu wykonano 10 pomiarów. Polegały one na zmniejszaniu napięcia na zasilaczu laboratoryjnym od wartości 15,5 V do 2 V, w krokach co 1,5 V, i rejestrowaniu wskazań obu mierników. Aby zapewnić jak największą dokładność odczytów, na bieżąco dostosowywano zakresy pomiarowe urządzeń (30V, 10V, 3V dla woltomierza i 200mA, 20mA dla amperomierza).
	
	\section{Wyniki pomiarów}
	\subsection{Część I: Wyznaczanie objętości walca}
	
	Wyniki pomiarów średnicy i wysokości walca zostały zawarte w tabeli poniżej.
	
	\begin{table}[H]
		\centering
		
		\begin{tabular}{|c|c|c|}
			\hline
			Numer pomiaru & d [mm] & H [mm] \\
			\hline
			1 & 11,74 & 47,1 \\
			2 & 11,76 & 47,1 \\
			3 & 11,76 & 47,1 \\
			4 & 11,74 & 47,1 \\
			5 & 11,75 & 47,4 \\
			6 & 11,76 & 47,1 \\
			7 & 11,77 & 47,1 \\
			8 & 11,77 & 46,8 \\
			9 & 11,77 & 47,1 \\
			10 & 11,77 & 47,1 \\
			11 & 11,76 & 47,1 \\
			12 & 11,77 & 47,1 \\
			13 & 11,77 & 47,1 \\
			14 & 11,76 & 47,0 \\
			15 & 11,76 & 47,1 \\
			\hline
		\end{tabular}
		\caption{Pomiary średnicy i wysokości walca}
		\label{pomiary_walec}
	\end{table}
	
	
	
	\subsection{Część II: Weryfikacja prawa Ohma}
	
	Wyniki pomiarów napięcia i natężenia wraz z niepewnościami umieszczono w tabeli poniżej.
	
	\begin{table}[H]
		\centering
		\begin{tabular}{|c|c|c|c|c|c|c|c|}
			\hline
			l.p. & U [V] & $Z_U$[V] & u(U) [V] & I [mA] & $Z_I$[mA] & $dgt_I$ [mA] & u(I) [mA] \\
			\hline
			1 & 11 & 30 & 0,17 & 29 & 200 & 0,1 & 0,259 \\
			\hline
			2 & 10 & 30 & 0,17 & 25,9 & 200 & 0,1 & 0,237 \\
			\hline
			3 & 9 & 30 & 0,17 & 23,3 & 200 & 0,1 & 0,219 \\
			\hline
			4 & 8 & 30 & 0,17 & 20,4 & 200 & 0,1 & 0,199 \\
			\hline
			5 & 6,8 & 10 & 0,058 & 17,49 & 20 & 0,01 & 0,056 \\
			\hline
			6 & 5,4 & 10 & 0,058 & 14,09 & 20 & 0,01 & 0,046 \\
			\hline
			7 & 4,6 & 10 & 0,058 & 12,09 & 20 & 0,01 & 0,041 \\
			\hline
			8 & 3,8 & 10 & 0,058 & 9,22 & 20 & 0,01 & 0,032 \\
			\hline
			9 & 2,6 & 3 & 0,017 & 6,51 & 20 & 0,01 & 0,025 \\
			\hline
			10 & 1,5 & 3 & 0,017 & 3,73 & 20 & 0,01 & 0,017 \\
			\hline
		\end{tabular}
		\caption{Wyniki pomiarów napięcia i natężenia}
		\label{wyniki_pomiarow_u_i}
	\end{table}
	
	Gdzie:
	\begin{conditions}
		Z_U & zakres woltomierza\\
		Z_I & zakres amperomierza\\
		dgt_I & wartość błędu wynikająca z rozdzielczości wyświetlacza na danym zakresie pomiarowym\\
	\end{conditions}
	
	\section{Opracowanie części I}
	\subsection{Niepewności typu A}
	
	W celu obliczenia niepewności pomiarowych metodą A należy skorzystać z danych zapisanych w \textbf{tabeli \ref{pomiary_walec}}. Aby to zrobić należy najpierw policzyć średnią arytmetyczną uzyskanych pomiarów \eqref{eq:srednia}.
	
	\begin{equation}
		\bar{d} = \frac{1}{15}*\sum_{i=1}^{15}d_i \approx 11,7607 mm
		\label{srednia_srednicy}
	\end{equation}
	
	\begin{equation}
		\bar{H} = \frac{1}{15}*\sum_{i=1}^{15}H_i \approx 47,0933 mm
		\label{srednia_wysokosci}
	\end{equation}
	
	\noindent Następnie można przystąpić do obliczenia niepewności pomiarowej metodą A korzystając ze wzoru \eqref{eq:metoda_a}.
	
	\begin{equation}
		u_A(d) = \sqrt{\frac{1}{15*14}\sum_{i=1}^{15}(d_i - \bar{d})^2} \approx 0,0026666666667 \approx 0,0027 mm
		\label{metoda_a_srednica}
	\end{equation}
	
	\begin{equation}
		u_A(H) = \sqrt{\frac{1}{15*14}\sum_{i=1}^{15}(H_i - \bar{H})^2} \approx 0,0300264433723 \approx 0,030 mm
		\label{metoda_a_wysokosc}
	\end{equation}
	
	\subsection{Niepewność typu B}
	Odczytując informacje zapisane na urządzeniach pomiarowych można ustalić niepewności wzorcowe i eksperymentalne.
	
	\begin{equation}
		\Delta d = 0,02 mm \implies \Delta d_e = 0,01 mm
	\end{equation}
	
	\begin{equation}
		\Delta H = 0,1 mm \implies \Delta H_e = 0,05 mm
	\end{equation}
	
	\noindent Korzystając z tych informacji i wzoru \eqref{metoda_b} można obliczyć niepewność typu B.
	
	\begin{equation}
		u_B(d) = \sqrt{\frac{\Delta d^2 + \Delta d_e^2}{3}} = \sqrt{\frac{0,02^2 + 0,01^2}{3}} \approx 0,01290994449 mm \approx 0,0129 mm
	\end{equation}
	
	\begin{equation}
		u_B(H) = \sqrt{\frac{\Delta H^2 + \Delta H_e^2}{3}} = \sqrt{\frac{0,1^2 + 0,05^2}{3}} \approx 0,06454972244 mm \approx 0,0645 mm
	\end{equation}
	
	\subsection{Niepewności całkowite}
	Mając niepewności typu A i B można przystąpić do liczenia niepewności całkowitych ze wzoru \eqref{eq:niepewnosc_calkowita}.
	
	\begin{equation}
		u(d) = \sqrt{u_A^2(d) + u_B^2(d)} \approx 0,0131824799581 mm \approx 0,013 mm
	\end{equation}
	
	\begin{equation}
		u(H) = \sqrt{u_A^2(H) + u_B^2(H)} \approx 0,0711916706411 mm \approx 0,071 mm
	\end{equation}
	
	\subsection{Niepewność objętości z prawa propagacji}
	Aby policzyć niepewność dla objętości należy skorzystać z prawa propagacji niepewności \eqref{eq:prawo_propagacji}.
	
	\begin{equation}
		V(d,H) = \frac{\pi d^2 H}{4} \approx 5115,79572542407 mm \approx 5116 mm
	\end{equation}
	
	\begin{equation}
		u(V) = \sqrt{\left(\frac{\partial V}{\partial d}*u(d)\right)^2 + \left(\frac{\partial V}{\partial H}*u(H)\right)^2} = \sqrt{\left(\frac{\pi d H}{2} * u(d)\right)^2 + \left(\frac{\pi d^2}{4} * u(H)\right)^2} \approx 13,83 mm^3 \approx 14 mm^3
	\end{equation}
	
	\subsection{Zapisanie wyników}
	\begin{equation}
		d = 11,760(13) \text{ mm}
	\end{equation}
	\begin{equation}
		H = 47,093(71) \text{ mm}
	\end{equation}
	\begin{equation}
		V = 5116(14) \text{ mm}^3
	\end{equation}
	
	\section{Opracowanie części II}
	
	\subsection{Niepewność pomiaru napięcia}
	W celu obliczenia niepewności wzorcowania należy skorzystać ze wzoru:
	\begin{equation}
		\Delta x = \frac{\text{klasa}*\text{zakres}}{100}
	\end{equation}
	
	\noindent Klasa woltomierza wynosi $k = 1$, a zatem dla zakresu 30V niepewność typu B będzie równa:
	\begin{equation}
		u_{30V}(U) = \frac{0,3}{\sqrt{3}} \approx 0,1732050808 V \approx 0,17 V
	\end{equation}
	Dla zakresu 10 V:
	\begin{equation}
		u_{10V}(U) = \frac{0,1}{\sqrt{3}} \approx 0,05773502692 V \approx 0,058 V
	\end{equation}
	Dla zakresu 3V:
	\begin{equation}
		u_{3V}(U) = \frac{0,03}{\sqrt{3}} \approx 0,01732050808 V \approx 0,017 V
	\end{equation}
	
	\subsection{Niepewności pomiaru natężenia}
	
	Zgodnie z instrukcją do miernika uniwersalnego M-3800, niepewność wzorcowania dla zakresu 200 mA wyraża się wzorem:
	\begin{equation}
		\Delta I = 1,2\%\text{rdg} + 1\text{dgt}
	\end{equation}
	
	\noindent Natomiast dla zakresu 20 mA:
	\begin{equation}
		\Delta I = 0,5\%\text{rdg} + 1\text{dgt}
	\end{equation}
	
	\subsection{Regresja liniowa}
	
	Na podstawie wyników pomiarów zapisanych w \textbf{tabeli \ref{wyniki_pomiarow_u_i}} sporządzono wykres zależności napięcia od natężenia, zaznaczono linię trendu oraz umieszczono słupki błędów, w zależności od wyznaczonych niepewności.
	
	\begin{figure}[H]
		\centering
		\includegraphics[width=16cm]{images/zaleznosc_u_od_i.png}
		\caption{Wykres zależności napięcia od natężenia}
		\label{fig:zaleznosc_u_od_i}
	\end{figure}
	
	\noindent Korzystając z funkcji REGLINP (ang. LINEST) w arkuszu kalkulacyjnym, można wyznaczyć poszukiwany opór i jego niepewność $R$.
	
	\begin{equation}
		R \approx 0,385897181393519\, k\Omega \approx 385,9\,\Omega
	\end{equation}
	
	\begin{equation}
		u(R) \approx 0.00211297293616013 \,\frac{V}{mA} \approx 2,1 \,\Omega
	\end{equation}
	
	\noindent Następnie można podać wynik:
	
	\begin{equation}
		R = 385,9(21)\,\Omega
	\end{equation}
	
	\subsection{Propagacja niepewności oporu obliczona z jednego pomiaru}
	
	Niepewność oporu $R_1$ można policzyć korzystając ze wzoru \ref{eq:prawo_propagacji}, używając napięcia $U_1$ ,natężenia $I_1$ i ich niepewności, które zdefiniowano poniżej.
	
	\begin{align*}
		U_1 &= 11 \text{ V} \\
		u(U_1) &= 0,17 \text{ V} \\
		I_1 &= 29 \text{ mA} \\
		u(I_1) &= 0,259 \text{ mA}
	\end{align*}
	
	\noindent Po zdefiniowaniu tych wartości można przejść do liczenia niepewności oporu:
	
	\begin{equation}
		u(R_1) = \sqrt{\left(\frac{\partial R_1}{\partial U_1}*u(U_1)\right)^2+\left(\frac{\partial R_1}{\partial I_1}*u(I_1)\right)^2} = \sqrt{\left(\frac{1}{I_1}*u(U_1)\right)^2+\left(-\frac{U_1}{I_1^2}*u(I_1)\right)^2}
	\end{equation}
	
	\begin{equation}
		u(R_1) \approx 0,006770518083\,\frac{V}{mA} \approx 6,8\,\Omega
	\end{equation}
	
	\noindent Następnie należy policzyć opór $R_1$ korzystając z $U_1$ i $I_1$
	
	\begin{equation}
		R_1 \approx \frac{U_1}{I_1} \approx 0,379310345\,\frac{V}{mA} \approx 379,3\,\Omega
	\end{equation}
	
	\noindent A zatem opór $R_1$ jest równy:
	\begin{equation}
		R_1 = 379,3(68)\,\Omega
	\end{equation}
	
	\section{Wnioski}
	\subsection{Część I: Wyznaczanie objętości walca}
	W ramach ćwiczenia wyznaczono objętość walca. Uzyskana wartość, wraz z obliczoną niepewnością rozszerzoną, wynosi:
	\begin{equation}
		V = 5116(14) \text{ mm}^3
	\end{equation}
	Analiza propagacji niepewności wykazała, że \textbf{dominującym źródłem niepewności całkowitej} był pomiar wysokości ($H$). Wynikało to bezpośrednio z niskiej rozdzielczości (błędu wzorcowania) użytej suwmiarki (0,1 mm), co przełożyło się na dużą niepewność pomiarową typu B.
	
	W celu znaczącej redukcji niepewności końcowej (potencjalnie do wartości jednocyfrowej), należałoby zastosować do pomiaru wysokości przyrząd o wyższej precyzji. Zwiększenie liczby pomiarów pozwoliłoby dodatkowo zredukować niepewność pomiarową typu A.
	
	\subsection{Część II: Weryfikacja prawa Ohma}
	Cel drugiej części, jakim była eksperymentalna weryfikacja prawa Ohma, został zrealizowany. Na podstawie zebranych danych pomiarowych ($I$, $U$) sporządzono wykres charakterystyki prądowo-napięciowej.
	
	Zastosowanie metody najmniejszych kwadratów (regresji liniowej) \textbf{potwierdziło zależność liniową} między napięciem a natężeniem prądu.
	
	Na podstawie regresji liniowej wyznaczono wartość oporu na:
	\begin{equation}
		R = 385,9(21)\,\Omega
	\end{equation}
	Dodatkowo, obliczono opór metodą propagacji niepewności dla pojedynczego pomiaru (Sekcja 6.1.4), uzyskując.

	\begin{equation}
		R_1 = 379,3(68)\,\Omega
	\end{equation}
	
	 Wynik ten jest zgodny z wartością z regresji, jednak jego niepewność jest ponad trzykrotnie większa. Dowodzi to statystycznej wyższości metody najmniejszych kwadratów, która skutecznie uśrednia błędy losowe ze wszystkich punktów pomiarowych.\\
	
	Obserwowane, niewielkie odchylenia punktów pomiarowych od wyznaczonej prostej regresji, można przypisać dwóm głównym czynnikom. Pierwszym są błędy wzorcowania użytej aparatury (woltomierza i amperomierza). Drugim, potencjalnie istotnym czynnikiem, mogły być \textbf{zmiany rezystancji opornika} spowodowane jego nagrzewaniem się. W celu dalszego uściślenia wyników należałoby zastosować mierniki o wyższej dokładności oraz zadbać o stabilizację termiczną badanego elementu.
	\section{Bibliografia}
	% Nie dotykać tego, bo nagłówek się nie numeruje normalnie
	\printbibliography[heading=none]
	
\end{document}